\documentclass[a4paper,11pt]{article}
\usepackage{pagecolor,lipsum}
\usepackage{hyperref}
\definecolor{LinkColor}{HTML}{FBF1C7}
\usepackage{xpatch}
\usepackage{fdsymbol}
\hypersetup{
    colorlinks,
    linkcolor={LinkColor},
    citecolor={LinkColor},
    urlcolor={LinkColor}
}
\definecolor{BlackHaf}{HTML}{101213}
\definecolor{WhiteHaf}{HTML}{EEEEEE}
\definecolor{LightOrangeHaf}{HTML}{C5A782}
\definecolor{OrangeHaf}{HTML}{DA8548}
\newcommand{\e}{\'{e}}
\makeatletter
\newcommand{\globalcolor}[1]{%
  \color{#1}\global\let\default@color\current@color
}
\makeatother
\AtBeginDocument{\globalcolor{WhiteHaf}}
\title{\color{OrangeHaf} \Huge Liste des Master info en France\\2021}
\date{}
\begin{document}
\pagecolor{BlackHaf}
    \maketitle
    \section{Introduction}
        Il s'agit de la liste complete de tout les
        \\
        \\
        \\
    \section{Liste}
    \begin{enumerate}
        \item [\color{LightOrangeHaf} Aix-Marseille Université:] Du 01/04 au 18/06 
                \\Capacité d'accueil : 120
                \\ Le \textbf{master informatique} offre une palette de six parcours adaptés à plusieurs secteurs de l'informatique. Les diplômés auront donc accès à une grande diversité de métiers et de carrières.
                \begin{itemize}
                    \item Parcours : Fiabilité et sécurité informatique (FSI)
                    \item Parcours : Ingénierie du logiciel et des données (ILD)
                    \item Parcours : Intelligence artificielle et apprentissage automatique (IAAA)
                    \item Parcours : Informatique et mathématiques discrètes (IMD)
                    \item Parcours : Géométrie et informatique graphique (GIG)
                \end{itemize}
                $Lien$: \href{https://informatique-sciences.univ-amu.fr/master-informatique}{Site Université}\\
                remarque: Beaucoup de choix, le parcours FSI semble interresant, mais la faiblesse du cours de L2 sur les r\e seaux et un souccis\dots
\\
        \item [\color{LightOrangeHaf} Avignon Université] Du 05/03 au 27/08 
                \\Capacité d'accueil : 100
                \\Le \textbf{Master Informatique }correspond à une formation de très haut niveau associant compétences théoriques et pratiques, bases scientifiques et stages en entreprise il se décline en trois parcours-types:
            \begin{itemize}
                \item Ingénierie du Logiciel de la SociétE Numérique (ILSEN)
                \item Systèmes Informatiques Communicants : réseaux, services et sécurité
                \item Intelligence Artificielle (IA)
            \end{itemize}
                $Lien$: \href{https://ceri.univ-avignon.fr/formations/master-informatique/}{Site Université}\\
                remarque: Rien de tres remarquable mais il semblerait qu'elle possede un taux de pression faible, a v\e rifier
\\
        \item [\color{LightOrangeHaf} La Rochelle Université] Du 19/04 au 17/05 
                \\Capacité d'accueil : 60
                \\À l’issue du \textbf{master Informatique}, vous saurez appréhender parfaitement l’organisation des flux numériques ainsi que la mise en place d’outils d’exploitation du patrimoine immatériel d’une entreprise ou d’une collectivité. De fait, vous pourrez traiter des problématiques de gestion et d’analyse des données, concevoir des systèmes d’information mais aussi exploiter et valoriser des contenus numériques.
                \begin{itemize}
                    \item parcours Données
                    \item parcours Architecte logiciel
                \end{itemize}
                $Lien$: \href{https://sciences.univ-larochelle.fr/master-informatique}{Site Université}\\
                \\remarque: petite université, peut'etre plus petite qu toulon, Incroyable.  M'enfin a l'aire sympa mais malheureusement aucun sujet $vraiment$ interresant.
\\
        \item [\color{LightOrangeHaf}Université de Lille] Du 19/04/2021 au 29/05/2021
                \\Capacité d'accueil : 140
                \\Le master mention Informatique de l’Université de Lille propose une formation de pointe pour les étudiants qui ciblent un poste de cadre dans le secteur des sciences de l’information et des technologies des communications. Ce master constitue une poursuite d’études naturelle pour les étudiants titulaires d’une Licence Informatique généraliste. Cette mention propose notamment 6 parcours répartis sur 2 années qui permettent aux étudiants d’approfondir leurs compétences dans des domaines d’actualités. 
                \begin{itemize}
                    \item Cloud computing
                    \item E-services
                    \item Genie logiciel
                    \item Internet des objets
                    \item Machine learning
                    \item Réalité virtuelle et augmentée
                \end{itemize}
                $Lien$: \href{https://sciences-technologies.univ-lille.fr/informatique/formation/master-informatique}{Site Université}\\
                \\remarque: Choix vaste est original, Le cloud computing est vraiment interresant mais peut de place(16 places), le genie logiciel me semble etre le meilleur que j'ai vue pour l'instant. En resumer Lille semble vraiment $cool$.
\\
        \item [\color{LightOrangeHaf}Université de Grenoble] Du 20/03 au 15/05
                \\Capacité d'accueil : 102
                \\L’objectif du master est de réaliser une formation de haut niveau en informatique pour des métiers de l’ingénierie logicielle et matérielle et de la recherche en informatique. La formation couvre un large spectre qui va de l’ingénierie logicielle et matérielle à l’informatique théorique en passant par plusieurs domaines, à l’interface des mathématiques et de l’informatique comme la sécurité informatique ou l’optimisation.
                \begin{itemize}
                    \item Compétences Complémentaires en Informatique (CCI) 
                    \item Cybersecurity
                    \item Master of Science in Informatics at Grenoble (MoSIG)
                    \item Operations Research, Combinatorics and Optimization (ORCO) 
                    \item Génie informatique (classique ou alternance) 2e année
                    \item Cybersécurité et informatique légale
                \end{itemize}
                $Lien$: \href{https://im2ag.univ-grenoble-alpes.fr/menu-principal/formation/offre-de-formation/master-informatique/}{Site Université}\\
                remarque: Grosse université, plein de chose interresant et international. Mais putain qu'elle bordel leurs site\ldots
\\
        \item [\color{LightOrangeHaf}Université de Dunkerque] Du 15/03 au 30/06
                Capacité d'accueil : 37
                \\\textbf{Master informatique} uniquement en alternance, la difficulter est donc de trouver une entreprise qui nous accepte. Autrement la formation de "$haut$ $niveaux$" encores possede deux option:
                \begin{itemize}
                    \item Ingénierie du logiciel libre
                    \item Web et Science de données
                \end{itemize}
                $Lien$: \href{https://www.univ-littoral.fr/formation/offre-de-formation/masters/master-informatique-ingenierie-des-systemes-informatiques-distribues/}{Web et Science de données}, \href{https://www.univ-littoral.fr/formation/offre-de-formation/masters/master-informatique-ingenierie-du-logiciel-libre/}{Ingénierie du logiciel libre}\\
                remarque:La partie logicielle libre semble tres interresant, le probleme et l'alternance.
\\
        \item [\color{LightOrangeHaf}Université de Vanne] Du 15/03 au 30/06
                \\Capacité d'accueil : 87
                \\La formation se déroule sur deux années. La première année constitue un tronc commun permettant à l'étudiant de consolider ses connaissances généralistes en informatique et de construire un socle méthodologique rigoureux. Les enseignements de seconde année sont spécialisés en fonction du parcours choisi.
                \begin{itemize}
                    \item Applications Interactives et Données Numériques
                    \item Systemes et Applications pour l'Informatique Mobile
                \end{itemize}
                $Lien$: \href{http://www-informatique.univ-ubs.fr/master-info/index.shtml}{Site Université}
                \\remarque: Whaou ca a l'air tres nul \ldots a mettre sur ecandidat doit y avoir plein de places.
\\
        \item [\color{LightOrangeHaf}Université de Clermont] Du 10/03 au 15/04
                \\Capacité d'accueil : 60
                \\Le master mention Informatique est une formation de deux ans orientée vers l'informatique décisionnelle et logicielle, la programmation 3D, la gestion des données image et les systèmes interactifs et immersifs 3D. Il s’agit d’un master indifférencié, à vocation professionnelle et recherche. La première année (M1) est constituée d'un tronc commun dédié aux enseignements des fondamentaux en informatique ainsi que des enseignements complémentaires et de spécialisation. La seconde année (M2) offre une spécialisation dans un des parcours:
                \begin{itemize}
                    \item Système d'information et aide à la décision
                    \item Génie logiciel, intégration d’applications
                    \item Développement, 3D et Intelligence Artificielle
                    \item International of Computer Science
                \end{itemize}
                $Lien$: \href{https://www.uca.fr/formation/nos-formations/catalogue-des-formations/master-info}{Site Université}
                remarque: L'Auvergne\dots pratiques pour la lessive, le master lui semble tres professionnalisant et peut technique, peut d'interret pour moi.
\\
        \item [\color{LightOrangeHaf}Université  Côte d'Azur] Du 12/04 au 27/05
                \\Capacité d'accueil : 90
                \\Le Master Informatique est une formation en 2 ans à la fois scientifique et technique. Il forme des cadres de l’industrie et de la recherche capables de maîtriser les concepts et outils informatiques les plus récents et d’anticiper les évolutions technologiques futures.Il possede 3 parcours:
                \begin{itemize}
                    \item EIT Digital Options Data Science, FinTech, Autonomous System
                    \item Informatique et interactions
                    \item Ingénierie informatique
                \end{itemize}
                $Lien$: \href{https://univ-cotedazur.fr/offre-de-formation/master-informatique-1}{Site Université}
                \\remarque: Le premier parcours ce concentre sur la recherche,nope. Ingénierie informatique a l'air sympa mais beaucoup de matiere BS a mon gout.
\\
        \item [\color{LightOrangeHaf}Université Anger] Du 26/04 au 21/05
                \\Capacité d'accueil : 40
                \\L’objectif du master est de former des informaticiens polyvalents capables d’appréhender complètement le cycle du développement logiciel depuis la définition des besoins jusqu’au déploiement et la validation d’une application informatique. La première année de master est commune aux parcours de M2 et prépare à l'entrée dans ces parcours:
                \begin{itemize}
                    \item Intelligence Décisionnelle
                    \item Analyse, Conception et Développement Informatiques
                \end{itemize}
                $Lien$: \href{http://www.info.univ-angers.fr/dptinfo/}{Site Université}
                remarque: Le deuxieme parcours n'as pas l'air si mauvais, m'enfin rien de folichon non plus.
                \\



    \end{enumerate}
\end{document}