\documentclass[a4paper,11pt]{article}
\usepackage{pagecolor,lipsum}
\usepackage{hyperref}
\definecolor{LinkColor}{HTML}{FBF1C7}
\usepackage{xpatch}
\usepackage{fdsymbol}
\hypersetup{
    colorlinks,
    linkcolor={LinkColor},
    citecolor={LinkColor},
    urlcolor={LinkColor}
}
\definecolor{BlackHaf}{HTML}{101213}
\definecolor{WhiteHaf}{HTML}{EEEEEE}
\definecolor{LightOrangeHaf}{HTML}{C5A782}
\definecolor{OrangeHaf}{HTML}{DA8548}
\newcommand{\e}{\'{e}}
\makeatletter
\newcommand{\globalcolor}[1]{%
  \color{#1}\global\let\default@color\current@color
}
\makeatother
\AtBeginDocument{\globalcolor{WhiteHaf}}
\title{\color{OrangeHaf} \Huge Liste des Master info en France\\2021}
\date{}
\begin{document}
\pagecolor{BlackHaf}
    \maketitle
    \section{Introduction}
        Il s'agit de la liste complete de tout les il y a en tout 41 jusqu'a Toulon.
        \\
        \\
        \\
    \section{Liste}
    \begin{enumerate}
        \item [\color{LightOrangeHaf} Aix-Marseille Université:] Du 01/04 au 18/06 
                \\Capacité d'accueil : 120
                \\ Le \textbf{master informatique} offre une palette de six parcours adaptés à plusieurs secteurs de l'informatique. Les diplômés auront donc accès à une grande diversité de métiers et de carrières.
                \begin{itemize}
                    \item Parcours : Fiabilité et sécurité informatique (FSI)
                    \item Parcours : Ingénierie du logiciel et des données (ILD)
                    \item Parcours : Intelligence artificielle et apprentissage automatique (IAAA)
                    \item Parcours : Informatique et mathématiques discrètes (IMD)
                    \item Parcours : Géométrie et informatique graphique (GIG)
                \end{itemize}
                $Lien$: \href{https://informatique-sciences.univ-amu.fr/master-informatique}{Site Université}\\
                remarque: Beaucoup de choix, le parcours FSI semble interresant, mais la faiblesse du cours de L2 sur les r\e seaux et un souccis\dots
\\
        \item [\color{LightOrangeHaf} Avignon Université] Du 05/03 au 27/08 
                \\Capacité d'accueil : 100
                \\Le \textbf{Master Informatique }correspond à une formation de très haut niveau associant compétences théoriques et pratiques, bases scientifiques et stages en entreprise il se décline en trois parcours-types:
            \begin{itemize}
                \item Ingénierie du Logiciel de la SociétE Numérique (ILSEN)
                \item Systèmes Informatiques Communicants : réseaux, services et sécurité
                \item Intelligence Artificielle (IA)
            \end{itemize}
                $Lien$: \href{https://ceri.univ-avignon.fr/formations/master-informatique/}{Site Université}\\
                remarque: Rien de tres remarquable mais il semblerait qu'elle possede un taux de pression faible, a v\e rifier
\\
        \item [\color{LightOrangeHaf} La Rochelle Université] Du 19/04 au 17/05 
                \\Capacité d'accueil : 60
                \\À l’issue du \textbf{master Informatique}, vous saurez appréhender parfaitement l’organisation des flux numériques ainsi que la mise en place d’outils d’exploitation du patrimoine immatériel d’une entreprise ou d’une collectivité. De fait, vous pourrez traiter des problématiques de gestion et d’analyse des données, concevoir des systèmes d’information mais aussi exploiter et valoriser des contenus numériques.
                \begin{itemize}
                    \item parcours Données
                    \item parcours Architecte logiciel
                \end{itemize}
                $Lien$: \href{https://sciences.univ-larochelle.fr/master-informatique}{Site Université}\\
                \\remarque: petite université, peut'etre plus petite qu toulon, Incroyable.  M'enfin a l'aire sympa mais malheureusement aucun sujet $vraiment$ interresant.
\\
        \item [\color{LightOrangeHaf}Université de Lille] Du 19/04 au 29/05
                \\Capacité d'accueil : 140
                \\Le master mention Informatique de l’Université de Lille propose une formation de pointe pour les étudiants qui ciblent un poste de cadre dans le secteur des sciences de l’information et des technologies des communications. Ce master constitue une poursuite d’études naturelle pour les étudiants titulaires d’une Licence Informatique généraliste. Cette mention propose notamment 6 parcours répartis sur 2 années qui permettent aux étudiants d’approfondir leurs compétences dans des domaines d’actualités. 
                \begin{itemize}
                    \item Cloud computing
                    \item E-services
                    \item Genie logiciel
                    \item Internet des objets
                    \item Machine learning
                    \item Réalité virtuelle et augmentée
                \end{itemize}
                $Lien$: \href{https://sciences-technologies.univ-lille.fr/informatique/formation/master-informatique}{Site Université}\\
                \\remarque: Choix vaste est original, Le cloud computing est vraiment interresant mais peut de place(16 places), le genie logiciel me semble etre le meilleur que j'ai vue pour l'instant. En resumer Lille semble vraiment $cool$.
\\
        \item [\color{LightOrangeHaf}Université de Grenoble] Du 20/03 au 15/05
                \\Capacité d'accueil : 102
                \\L’objectif du master est de réaliser une formation de haut niveau en informatique pour des métiers de l’ingénierie logicielle et matérielle et de la recherche en informatique. La formation couvre un large spectre qui va de l’ingénierie logicielle et matérielle à l’informatique théorique en passant par plusieurs domaines, à l’interface des mathématiques et de l’informatique comme la sécurité informatique ou l’optimisation.
                \begin{itemize}
                    \item Compétences Complémentaires en Informatique (CCI) 
                    \item Cybersecurity
                    \item Master of Science in Informatics at Grenoble (MoSIG)
                    \item Operations Research, Combinatorics and Optimization (ORCO) 
                    \item Génie informatique (classique ou alternance) 2e année
                    \item Cybersécurité et informatique légale
                \end{itemize}
                $Lien$: \href{https://im2ag.univ-grenoble-alpes.fr/menu-principal/formation/offre-de-formation/master-informatique/}{Site Université}\\
                remarque: Grosse université, plein de chose interresant et international. Mais putain qu'elle bordel leurs site\ldots
\\
        \item [\color{LightOrangeHaf}Université de Dunkerque] Du 15/03 au 30/06
                Capacité d'accueil : 37
                \\\textbf{Master informatique} uniquement en alternance, la difficulter est donc de trouver une entreprise qui nous accepte. Autrement la formation de "$haut$ $niveaux$" encores possede deux option:
                \begin{itemize}
                    \item Ingénierie du logiciel libre
                    \item Web et Science de données
                \end{itemize}
                $Lien$: \href{https://www.univ-littoral.fr/formation/offre-de-formation/masters/master-informatique-ingenierie-des-systemes-informatiques-distribues/}{Web et Science de données}, \href{https://www.univ-littoral.fr/formation/offre-de-formation/masters/master-informatique-ingenierie-du-logiciel-libre/}{Ingénierie du logiciel libre}\\
                remarque:La partie logicielle libre semble tres interresant, le probleme et l'alternance.
\\
        \item [\color{LightOrangeHaf}Université de Vanne] Du 15/03 au 30/06
                \\Capacité d'accueil : 87
                \\La formation se déroule sur deux années. La première année constitue un tronc commun permettant à l'étudiant de consolider ses connaissances généralistes en informatique et de construire un socle méthodologique rigoureux. Les enseignements de seconde année sont spécialisés en fonction du parcours choisi.
                \begin{itemize}
                    \item Applications Interactives et Données Numériques
                    \item Systemes et Applications pour l'Informatique Mobile
                \end{itemize}
                $Lien$: \href{http://www-informatique.univ-ubs.fr/master-info/index.shtml}{Site Université}
                \\remarque: Whaou ca a l'air tres nul \ldots a mettre sur ecandidat doit y avoir plein de places.
\\
        \item [\color{LightOrangeHaf}Université de Clermont] Du 10/03 au 15/04
                \\Capacité d'accueil : 60
                \\Le master mention Informatique est une formation de deux ans orientée vers l'informatique décisionnelle et logicielle, la programmation 3D, la gestion des données image et les systèmes interactifs et immersifs 3D. Il s’agit d’un master indifférencié, à vocation professionnelle et recherche. La première année (M1) est constituée d'un tronc commun dédié aux enseignements des fondamentaux en informatique ainsi que des enseignements complémentaires et de spécialisation. La seconde année (M2) offre une spécialisation dans un des parcours:
                \begin{itemize}
                    \item Système d'information et aide à la décision
                    \item Génie logiciel, intégration d’applications
                    \item Développement, 3D et Intelligence Artificielle
                    \item International of Computer Science
                \end{itemize}
                $Lien$: \href{https://www.uca.fr/formation/nos-formations/catalogue-des-formations/master-info}{Site Université}
                remarque: L'Auvergne\dots pratiques pour la lessive, le master lui semble tres professionnalisant et peut technique, peut d'interret pour moi.
\\
        \item [\color{LightOrangeHaf}Université  Côte d'Azur] Du 12/04 au 27/05
                \\Capacité d'accueil : 90
                \\Le Master Informatique est une formation en 2 ans à la fois scientifique et technique. Il forme des cadres de l’industrie et de la recherche capables de maîtriser les concepts et outils informatiques les plus récents et d’anticiper les évolutions technologiques futures.Il possede 3 parcours:
                \begin{itemize}
                    \item EIT Digital Options Data Science, FinTech, Autonomous System
                    \item Informatique et interactions
                    \item Ingénierie informatique
                \end{itemize}
                $Lien$: \href{https://univ-cotedazur.fr/offre-de-formation/master-informatique-1}{Site Université}
                \\remarque: Le premier parcours ce concentre sur la recherche,nope. Ingénierie informatique a l'air sympa mais beaucoup de matiere BS a mon gout.
\\
        \item [\color{LightOrangeHaf}Université Anger] Du 26/04 au 21/05
                \\Capacité d'accueil : 40
                \\L’objectif du master est de former des informaticiens polyvalents capables d’appréhender complètement le cycle du développement logiciel depuis la définition des besoins jusqu’au déploiement et la validation d’une application informatique. La première année de master est commune aux parcours de M2 et prépare à l'entrée dans ces parcours:
                \begin{itemize}
                    \item Intelligence Décisionnelle
                    \item Analyse, Conception et Développement Informatiques
                \end{itemize}
                $Lien$: \href{http://www.info.univ-angers.fr/dptinfo/}{Site Université}
                \\remarque: Le deuxieme parcours n'as pas l'air si mauvais, m'enfin rien de folichon non plus.
\\
        \item [\color{LightOrangeHaf}Université de Artois] Du 01/04 au 17/05
                \\Capacité d'accueil : 45
                \\Le master d'informatique de l'Université d'Artois vise à former des informaticiens de niveau BAC+5 ayant une formation générale solide dans les principaux domaines de l'informatique leur permettant de s'adapter aux outils et méthodes, actuels et futurs, du métier d'informaticien.\\Trois parcours, en seconde année, permettent d'enrichir les compétences générales acquises en première année par des compétences spécifiques en:
                \begin{itemize}
                    \item ingénierie logicielle pour l'internet
                    \item ingénierie logicielle pour les jeux
                    \item intelligence artificielle
                \end{itemize}
                $Lien$: \href{http://informatique.univ-artois.fr/master/}{Site Université}
                \\remarque: En vrai, pas mal du tout les 3 choix sont complet, le premier m'interrese pas mal, le deuxieme plutot rare et le troisieme n'est pas pour moi, en plus les locaux sont magnifique.
\\
        \item [\color{LightOrangeHaf}Université d'Orléans] Du 07/05 au 30/09
                \\Capacité d'accueil : 60
                \\Le Master IMIS forme des informaticiens disposant non seulement de compétences solides en informatique générale, pratique et théorique, mais également d'une spécialisation dans la thématique des applications nomades. Il vise à former des cadres aptes à appréhender les différents problèmes que posent la conception et le développement de logiciels.
                $Lien$: \href{https://formation.univ-orleans.fr/fr/formation/offre-de-formation/master-lmd-XB/sciences-technologies-sante-STS/master-informatique-parcours-informatique-mobile-intelligente-et-securisee-imis-program-espm2mt-514-2-3-2-2-2-2-2-2.html?univorleans}{Site Université}
                \\remarque: Le site pue du cul, comme la formation d'ailleur\ldots
\\
        \item [\color{LightOrangeHaf}Université de Besancon] Du 19/05 au 06/09
                \\Capacité d'accueil : 145
                \\Le cœur du master Informatique, parcours Ingénierie Systèmes et Logiciels porte sur les techniques approfondies du développement logiciel, en phase avec les thématiques phares du département de recherche DISC de l'institut FEMTO-ST et de l'écosystème industriel de la formation. Ce parcours est le parcours standard de Master en présentiel; il est renforcé dans le parcours CMI (Cursus Master en Ingénierie, parcours) qui contient en plus quelques Unités d'Enseignement. Il est également réalisable en alternance.
                \\Peut etre deux formation mais difficile a dire il est marquer que la première ferme\dots
                \begin{itemize}
                    \item Ingénierie du Test et de la Validation Logiciels et systèmes
                    \item Ingénierie Systèmes et Logiciels
                \end{itemize}
                $Lien$: \href{http://formation.univ-fcomte.fr/master/informatique-ingenierie-systemes-et-logiciels}{Site Université}
                \\remarque:la premier sp\e \ a l'air sympa dommage qu'elle semble fermer (ou uniquement en distancielle)a $verifier$.
\\
        \item [\color{LightOrangeHaf}Université de Bordeaux] Du 26/04 au 17/05
                \\Capacité d'accueil : 150
                \\Le Master Informatique vise à former des informaticiens susceptibles d'exercer un métier d'ingénieur d'études ou d'ingénieur en développement, éventuellement spécialisés dans un domaine d'application en fonction des parcours choisis. Il prépare également à une insertion professionnelle dans une équipe de recherche publique ou privée, soit comme ingénieur, soit comme doctorant.
                \begin{itemize}
                    \item Systèmes mobiles autonomes communicants
                    \item Calcul haute performance
                    \item Réseaux de communications et internet
                    \item Vérification logicielle
                    \item Informatique fondamentale
                    \item Informatique pour l'image et le son
                    \item Génie logiciel 
                    \item Cryptologie et sécurité informatique
                    \item Image processing and computer vision 
                    \item Image processing and computer vision 
                \end{itemize}
                $Lien$: \href{https://www.u-bordeaux.fr/formation/2020/PRMA_68/informatique}{Site Université}
                \\remarque: Plein de belle choses, des stat (all\e luia), et pleins de choix ! La crypto et la verification logiciel en particulier on l'air interresant.
\\
        \item [\color{LightOrangeHaf}Université de Bourgogne] Du 26/04 au 17/05
                \\Capacité d'accueil : 70  \underline{mais de droit pour les license info !}
                \\Les  étudiants  ayant  acquis  à  la  fin  de  la  1ère  année  du  Master  Informatique  des  connaissances  approfondies théoriques  et  pratiques  dans  divers  domaines  informatiques  peuvent  être  recrutés  en  tant  que  responsables ou chargés d’études et développement dans tous les domaines des entreprises qui utilisent l’informatique.\\ la 2ème année du Master constitue la véritable année de spécialisation, avec le choix entre plusieurs parcours en informatique :
                \begin{itemize}
                    \item Bases de données -Intelligence artificielle
                    \item Image et Intelligenceartificielle.
                \end{itemize}
                $Lien$: \href{https://www.u-bourgogne.fr/images/stories/odf/master/ff-informatique-m1-2.pdf}{Site Université}
                \\remarque: il y un troisieme choix de sp\e \ de s\e curiter au Cameroune!
\\
        \item [\color{LightOrangeHaf}Université de Brest] Du 20/04 au 05/07
                \\Capacité d'accueil : 99
                \\Blabla \ldots Blabla \dots 6 formation:
                \begin{itemize}
                    \item Parcours Ingénierie du logiciel, applications aux données environnementales
                    \item Parcours Logiciels pour systèmes embarqués
                    \item Parcours Développement logiciel des systèmes d'information
                    \item Parcours Technologies de l'information et ingénierie du logiciel par alternance
                    \item Parcours International
                    \item Parcours Compétences complémentaires dans les services du numérique
                    \item Parcours Systèmes interactifs, intelligents et autonomes
                \end{itemize}
                $Lien$: \href{http://formations.univ-brest.fr/fr/index/sciences-technologies-sante-STS/master-XB/master-informatique-INRBV8Y3//parcours-ingenierie-du-logiciel-applications-aux-donnees-environnementales-INRBV93V.html}{Site Université}
                \\ remarque: rien de remarquable, le moin pire et et le premier.
\\
        \item [\color{LightOrangeHaf}Université de Caen] Du 26/04 au 02/09
                \\Capacité d'accueil : 70
                \\Blabla \ldots Blabla \dots 4 formation:
                \begin{itemize}
                    \item Parcours Décision \& optimisation (DOP)
                    \item Parcours Internet, données \& connaissances (IDC)
                    \item Parcours Image \& données multimédia (IDM)
                \end{itemize}
                $Lien$: \href{https://uniform.unicaen.fr/catalogue/index?mot-cle=info&composante=&type=M&domaine=STS&modalite=&fi=0&fc-hcp=0&contrat-pro=0&fa=0&s=trouver-sa-formation&r=&submit=Rechercher}{Site Université}
                \\ remarque:E-secure et pas mal mais porter sur le doctorat, tres peut pour moi\ldots Le dop et sans doute le meilleur choix.
\\
        \item [\color{LightOrangeHaf}Université de Corse] Du 26/04 au 16/06
                \\Capacité d'accueil : 24
                \\Le diplômé de Master Informatique est un professionnel de niveau bac+5 possédant une large culture informatique et des compétences plus spécifiques liées aux technologies de conception et de développement d'applications web, mobile et desktop. Un Développeur Full Stack possède des compétences multiples. Il maitrise un grand nombre de technologies côté serveur (Back End) et côté client (Front End). Il est apte à concevoir et mettre en oeuvre des architectures adaptées à tout type d'application. Il peut superviser des projets de développement en couvrant tous les aspects allant du choix des technologies et Frameworks, à l'organisation et à la structuration du code, en passant par la prise en compte de l'expérience utilisateur, l'optimisation du référencement ou la sécurité, tout en proposant un mode de gestion de projet adapté, agile ou autre. 
                \\$Lien$: \href{https://applisweb.universita.corsica/devu/fiches_diplomes/fiches_diplomes-front/index-front.php?rbk=2&profil=&id_site=1&acces=ok&id_art=389&id_rub=162&id_fiche=CST_M_INFORMATIQUE_DEVFULLSTACK}{Site Université}
                \\ remarque: Un des rare master fullstack de France, pas loins\dots le dev web m'interrese pas mais c'est on jamais.
\\
        \item [\color{LightOrangeHaf}Université de La Réunion] Du 01/04 au 03/05
                \\Capacité d'accueil : 20
                \\L'objectif essentiel du Master Informatique est de former des cadres techniques du secteur informatique répondant à la demande socio-professionnelle nationale et locale telle qu'exprimée dans le cadre du plan de gestion prévisionnelle des emplois et des compétences de la filière TIC réunionnaise. Eventuellement, une poursuite d'études en doctorat est envisageable, en particulier au LIM, le laboratoire d'informatique et de mathématiques qui adosse la formation.
                \begin{itemize}
                    \item spécialité métier de l'enseignement
                    \item spécialité Ingénierie informatique
                    \item spécialité Services Informatiques Connectés et Sécurisés : \underline{formation en alternance}.
                \end{itemize}
                $Lien$: \href{https://sciences.univ-reunion.fr/formation/offre-de-formations-2020-2024/masters-2020-2024/informatique}{Site Université}
                \\ remarque: Metier de l'enseignment, asser unique\dots m'enfin choix deux a priori.
\\
        \item [\color{LightOrangeHaf}Université de Limoges] Du 15/03 au 16/05
                \\Capacité d'accueil : 55
                \\Blabla \ldots Blabla \dots 2 formation:
                \begin{itemize}
                    \item cryptis - Sécurité de l’Information et Cryptologie
                    \item isicg - Informatique, synthèse d’images et conception graphique
                \end{itemize}
                $Lien$: \href{http://www.cryptis.fr/index.php}{Site Université}
                \\ remarque: le parcours cryptis a l'air un peut viellos mais reconue donc pourquoi pas ?
\\
        \item [\color{LightOrangeHaf}Université de Lorraine] Du 15/03 au 30/07
                \\Capacité d'accueil : 140
                \\L’objectif du master informatique de l’Université de Lorraine est de proposer une formation avancée, fondamentale et appliquée, qui garantit une solide culture de base en informatique complétée par des enseignements spécialisés en fonction du projet professionnel de l’étudiant.
                \begin{itemize}
                    \item Apprentissage, Vision, Robotique (AVR)
                    \item Sécurité Informatique, Réseaux et Architectures Virtuelles (SIRAV) 
                    \item Ingénierie Logicielle (IL) 
                    \item Systèmes d’Information Décisionnelle (SID)
                    \item Génie Informatique (GI)
                \end{itemize}
                $Lien$: \href{http://licence-master-informatique.formation.univ-lorraine.fr/master/}{Site Université}
                \\ remarque: le parcours SIS a l'aire vraiment interresant ! (Beaucoup de chose sur le site il faudras y retourner)
\\
        \item [\color{LightOrangeHaf}Université de Montpellier] Du 15/03 au 30/07
                \\Capacité d'accueil : 160
                \\Le master informatique est structuré en cinq parcours clairement identifiés à finalité indifférenciée, recherche et professionnelle. Ces cinq parcours visent à former des cadres en informatique avec des compétences en architecture, conception des logiciels et des systèmes d’information, gestion et exploitation des données, modélisation et optimisation combinatoire, technologies du web et des réseaux, traitement d’images, notamment 3D, et traitement des langues et du langage naturel. Ces cinq parcours sont : 
                \begin{itemize}
                    \item Algo (Algorithmique)
                    \item GL (Génie Logiciel)
                    \item Imagine (image et jeux vidéo)
                    \item IASD (Intelligence Artificielle et Sciences des Données)
                    \item ICo (Intégration de Compétences)
                \end{itemize}
                $Lien$: \href{https://formations.umontpellier.fr/fr/formations/master-lmd-XB/master-informatique-ME154.html}{Site Université}
                \\ remarque: rien de remarquable\ldots a la limite le GL.
\\
        \item [\color{LightOrangeHaf}Université de Mulhouse] Du 19/04 au 29/05
                \\Capacité d'accueil : 20
                \\Le parcours IM du Master Informatique de l’Université de Haute-Alsace est co-accrédité avec l’Université de Strasbourg. Il est dispensé à Mulhouse sur le campus de l’Ilberg de l’Université de Haute-Alsace.  Il est destiné aux étudiants désirant développer leurs connaissances informatiques avec un axe fort sur la mobilité et la programmation.
                $Lien$: \href{https://www.fst.uha.fr/index.php/formations/masters/master-informatique-et-mobilite/}{Site Université}
                \\Remarque: Peut'etre le master informatique le \textbf{plus nul de France}, apres Toulon of course.
\\
        \item [\color{LightOrangeHaf}Université de Nante] Du 26/04 au 17/05
                \\Capacité d'accueil : 81
                \\Ce Master Informatique englobe trois secteurs (logiciel, IA et données, IA et optimisation) qui se déclinent en 5 parcours proposés sur 2 ans. Il est co-accrédité avec l'IMT Atlantique (Mines Nantes). La moité des enseignements est en socle commun en M1:
                \begin{itemize}
                    \item Architectures Logicielles
                    \item Apprentissage et Traitement Automatique de la Langue
                    \item Data Sciences
                    \item Visual Computing
                    \item Optimisation en recherche opérationnelle
                \end{itemize}
                $Lien$: \href{https://sciences-techniques.univ-nantes.fr/formations/masters/master-informatique}{Site Université}
                \\remarque: Optimisation et plutot decevant au final du bulshit sur l'IA encores, m'enfin l'architecture logiciel et pas mal.
\\
        \item [\color{LightOrangeHaf}Université de Paris] Du 15/04 au 07/07
                \\Capacité d'accueil : 94
                \\Blabla \ldots Blabla \dots 2 formation:
                \begin{itemize}
                    \item Cybersécurité
                    \item Vision et machine intelligente
                    \item Machine learning pour la science des données
                    \item Intelligence artificielle distribuée
                    \item Cybersécurité et esanté
                    \item Apprentissage machine pour la sciences des données
                    \item Génie informatique en Alternance
                \end{itemize}
                    $Lien$: \href{https://odf.u-paris.fr/fr/offre-de-formation/master-XB/sciences-technologies-sante-STS/informatique-K2NDIF4R.html}{Site Université}
                    :L'agternance a l'aire \textbf{G\'{E}NI} e Informatique en \textbf{AL}ternance (c'est le vrai nom du master putain) 
\\
        \item [\color{LightOrangeHaf}Université de Pau] Du 22/04 au 02/05
                \\Capacité d'accueil : 59
                \\Son offre d’enseignement est orientée à la fois vers la recherche et vers les entreprises. La formation a comme objectifs de former des ingénieurs informaticiens, répondant à des besoins et à des métiers bien identifiés, pour un marché reconnu déficitaire depuis de nombreuses années. Elle irrigue l’ensemble du territoire et plus particulièrement l’Aquitaine :
                    \begin{itemize}
                        \item Parcours Big Data
                        \item Parcours Systèmes Informatiques pour le Génie de la Logistique Industrielle et des Services (SIGLIS)
                        \item Parcours technologie de l'internet (TI)
                    \end{itemize}
                    $Lien$: \href{https://formation.univ-pau.fr/fr/catalogue/sciences-technologies-sante-STS/master-14/master-informatique-79_1.html}{Site Université}
                    \\ remarque: il essaye d'etre original dommage qu'il ne soit pas interresant, le parcours TI a l'air agr\e able
\\
        \item [\color{LightOrangeHaf}Université de Picardie] Du  au 
        \\ Rien en M1 ? a verifier\ldots
\\
        \item [\color{LightOrangeHaf}Université de Poitiers] Du 26/04 au 17/05
                    \\Capacité d'accueil : 70
                    \\Le master Informatique a pour objectif de contribuer à répondre aux très importants besoins sociétaux en informaticiens, que ce soit en ingénierie, en recherche et en formation.  Il prépare les étudiants à une insertion professionnelle directe ou à une poursuite d'études en doctorat. La 2ème année de master est ouverte à l'alternance (contrats de professionnalisation).  Le master se décline en 3 parcours :
                    \begin{itemize}
                        \item Conception logicielle
                        \item Gestion et analyse de données
                        \item Informatique embarquée
                    \end{itemize}
                    $Lien$: \href{http://formations.univ-poitiers.fr/fr/index/master-XB/master-XB/master-informatique-JAJEEY0P.html}{Site Université}
                    \\ remarque: rien de fou encores, informatique embarquée a la limite.
\\
        \item [\color{LightOrangeHaf}Université de Reims] Du 22/04 au 17/05
                    \\Capacité d'accueil : 32
                    \\Les enseignements sont dispensés sous forme de cours magistraux, de travaux dirigés et de travaux pratiques, l’accent étant mis sur les projets individuels et collectifs, les stages, ainsi que sur le contrôle continu.
                    $Lien$: \href{https://www.univ-reims.fr/formation/catalogue-de-formation/master-informatique,23515,38949.html?args=R9qFsCnMmKDtxCa17YTDkHVqaqbfYRXwwTnCVt2witCDUIiVoUdkeMDp%2AXGEGm2SMIhvMbuZ3_kOrRxvJlk6dOorIryuNioRCyFFyPAvhl9tCdwYdtHRrwAvNC1tDg_H&formation_id=198}{Site Université}
                    \\remarque: un peu nul\dots moins que toulon mais quand meme.
\\
        \item [\color{LightOrangeHaf}Université de Rennes] Du 26/04 au 17/05
                    \\Capacité d'accueil : 190
                    \\Le Master Informatique de l'ISTIC est un master qui prépare aux carrières d'ingénieur, de manager et de chercheur en informatique.
Il offre des débouchés régionaux importants, au coeur de la technopole Rennes Atalante qui regroupe plus de 300 entreprises, laboratoires et écoles des secteurs informatique, télécoms et médias, mais aussi en France et à l'international. 
                    \begin{itemize}
                        \item Ingénierie Logicielle (IL)
                        \item Cybersécurité
                        \item Cloud \& Réseaux
                        \item Science Informatique
                        \item Compétences Complémentaires pour les services du Numérique
                    \end{itemize}
                    $Lien$: \href{https://istic.univ-rennes1.fr/master-informatique-premiere-annee}{Site Université}
                    \\remarque: meilleur master cyber de France surement l'ideal pour moi, apres le cloud et reseaux a l'aire vraiment cool aussi.
\\
        \item [\color{LightOrangeHaf}Université de Rouen] Du 26/04 au 17/05
                    \\Capacité d'accueil : 90
                    \\Pas sur qu'il y a bien le M1 les information se contredisent m'enfin 3 formation:
                    \begin{itemize}
                        \item Génie de l'informatique logicielle
                        \item Informatique théorique et applications 
                        \item Sécurité des systèmes informatiques
                    \end{itemize}
                    $Lien$: \href{http://sciences-techniques.univ-rouen.fr/offre-de-formation-les-masters-332182.kjsp?RH=1378317039388&RF=1378324428882}{Site Université}
                    \\ remarque: terriblement banal, m'enfin securite a priori
\\
        \item [\color{LightOrangeHaf}Université de Strasbourg] Du 16/03 au 17/05
                    \\Capacité d'accueil : 155
                    \begin{itemize}
                        \item  Gestion de projets informatiques (GPI)
                        \item Image et 3D (I3D)
                        \item Science et ingénierie des réseaux, de l'Internet et des systèmes (SIRIS)
                        \item Science et ingénierie du logiciel (SIL)
                        \item Sciences des données et systèmes complexes (SDSC)
                        \item Data Sciences and Artificial Intelligence (UFAZ)
                    \end{itemize}
                    $Lien$: \href{https://www.unistra.fr/index.php?id=27887&tx_unistrarof_pi1%5Brof-program%5D=ME192&cHash=c1efbb9365a532ad89a9ceaed0405dc4#data-rof-tab-presentation}{Site Université}
                    \\ remarque: le parcours Siris a l'air cool mais j'ai de sale note i42, a voir donc\ldots
\\
        \item [\color{LightOrangeHaf}TOULON] Du 29/03 au 31/05
                    \\Capacité d'accueil : 20
                    \\prout
\\
        \item [\color{LightOrangeHaf}Université de Tours] Du 26/03 au 16/05
                    \\Capacité d'accueil : 25
                    \\des experts de la gestion et de l’analyse des données massives et complexes, sur la base de compétences en intelligence décisionnelle (BI), intelligence artificielle (IA) et pour répondre à des besoins économiques et sociétaux forts.
                    $Lien$: \href{https://www.univ-tours.fr/formations/master-sciences-technologies-sante-mention-informatique-parcours-big-data-management-and-analytics}{Site Université}
                    \\ remarque: Que dire c'est pas nul, mais pas folichon.
\\
        \item [\color{LightOrangeHaf}universite des Antilles] Du 26/03 au 16/05
                    \\Capacité d'accueil : 20
                    \\Permettre d’acquérir des techniques liées aux applications logicielles. Etudier les concepts fondamentaux nécessaires pour couvrir un spectre complet de compétences et une compréhension approfondie des mécanismes sous-jacents pour aborder correctement ce domaine en évolution constante.
                    $Lien$: \href{http://www.univ-antilles.fr/offre-formations/offre-formation-luniversite-des-antilles}{Site Université}
                    \\ remarque: Peut concurencer TOULON
\\
        \item [\color{LightOrangeHaf}Université du Mans] Du 26/03 au 17/05
                    \\Capacité d'accueil : 30
                    \\Le master Informatique offre une formation généraliste en informatique la première année, complétée par une spécialisation sur des thématiques de pointe la deuxième année.  Après la première année, formée d’un tronc commun général, les étudiants choisissent l’un des trois parcours de seconde année:
                    \begin{itemize}
                        \item Apprentissage et Fouille de Données
                        \item Apprentissage et traitement automatique des langues
                    \end{itemize}
                    $Lien$: \href{http://www.univ-lemans.fr/fr/formation/catalogue-des-formations/master-lmd-MLMD/sciences-technologies-sante-0004/master-informatique-IXQ1QBYN.html}{Site Université}
                    \\ remarque: encore de l'ia le premier parcours et plus interresant.
\\
        \item [\color{LightOrangeHaf}Université Gustave Eiffel] Du 03/02 au 16/05
                    \\Capacité d'accueil : 75
                    \\Les métiers visés par ce master concernent la gestion de projet informatique, le développement informatique (conception de systèmes et développement d’applications), la recherche et le développement… La poursuite d'études se fait généralement dans un des parcours-type de 2e année de master de notre université:
                    \begin{itemize}
                        \item image numérique 
                        \item développement logiciel 
                        \item internet des objets
                        \item intelligence artificielle
                    \end{itemize}
                    $Lien$: \href{https://formations.univ-gustave-eiffel.fr/index.php?id=1941&tx_agof_brochure%5Bbrochure%5D=693&tx_agof_brochure%5Bcontroller%5D=Brochure&tx_agof_brochure%5Baction%5D=show&cHash=34eb0a5b9846ddb9befcec49d7ffaa2a}{Site Université}
                    \\ remarque: Environs de Paris a eviter donc, le developement surement.
\\
        \item [\color{LightOrangeHaf}Université Saint-Etienne] Du 05/04 au 11/05
                    \\Capacité d'accueil : 90
                    \\Blabla \ldots Blabla \dots 3 formation:
                    \begin{itemize}
                        \item Données et Systèmes Connectés
                        \item Cyber-Physical Social Systems
                        \item Machine Learning and Data Mining
                    \end{itemize}
                    $Lien$: \href{https://depinfo.univ-st-etienne.fr/Master/index.php}{Site Université}
                    \\ remarque: le cyber bidule sur les smart city et original, peut'etre interresant.
\\
        \item [\color{LightOrangeHaf}Universite le Havre] Du 26/04 au 17/05
                    \\Capacité d'accueil : 36
                    \\Ce master informatique de l’université du Havre forme des cadres de haut niveau, ingénieurs et chercheurs possédant une solide culture informatique.  Les compétences acquises, en particulier, concernent le Web, l’internet des objets, la sécurité, la programmation mobile et le Big Data. Les jeunes diplômés sont capables d’exercer leur profession dans des industries de haute technologie, mais aussi d’apporter leur expertise dans des PME et des TPE.
                    $Lien$: \href{https://www.univ-lehavre.fr/spip.php?formation22}{Site Université}
                    \\ remarque: S tier pour shit tier, Toulon/20
\\
        \item [\color{LightOrangeHaf}Université Lumière Lyon 2] Du 06/04 au 02/05
                    \\Capacité d'accueil : 72
                    \\La première année du master informatique de l'Université Lyon 2 est commune aux différents parcours M2 informatique que propose l’institut de la communication. Le M1 informatique permet de donner aux étudiants une formation généraliste solide sur les aspects théoriques, méthodologiques et pratiques de l'informatique dans les domaines des bases de données, des systèmes d'information, de la programmation avancée, de la programmation web, de la fouille de données, de la statistique, … et des réseaux.
                    \begin{itemize}
                        \item programmation et développement de jeux vidéo (gamagora) 
                        \item visualisation et conception infographiques en ligne (cim - vciel) 
                        \item data mining
                        \item conception et intégration multimédia 
                        \item statistique et informatique statistique et Informatique pour la science des données sise 
                        \item organisation et protection des systèmes d'information en entreprise (opsie) - fi 
                    \end{itemize}
                    $Lien$: \href{https://www.univ-lyon2.fr/master-1-informatique-1}{Site Université}
                    \\ remarque: le jeux video a l'air sympa, mais l'OPSIE aussi\ldots
\\
        \item [\color{LightOrangeHaf}Université Lyon-I] Du 26/04 au 23/05
                    \\Capacité d'accueil : 130
                    \\La composante Informatique de l’Université Claude Bernard Lyon 1 attache une grande importance à la mission de formation de futurs cadres dans le respect des orientations définies par le Ministère chargé de l’Enseignement Supérieur et de la Recherche et dans le cadre de la construction de l’Espace Européen de l’Enseignement Supérieur. Neuf mentions sont proposées en Informatique à l'Université Lyon 1:
                    \begin{itemize}
                        \item AEU Préparation CAPES Informatique
                        \item Data science
                        \item Image, développement et technologie 3D
                        \item Informatique fondamentale
                        \item Ingénierie Technico-commerciale
                        \item Intelligence artificielle
                        \item Systèmes, réseaux et infrastructures virtuelles
                        \item Technologies de l'information et web
                    \end{itemize}
                    $Lien$: \href{http://offre-de-formations.univ-lyon1.fr/mention-52/informatique.html}{Site Université}
                    \\ remarque: le systemes a l'air sympa, pour le reste cela semble solide.
\\
        \item [\color{LightOrangeHaf}Université Paris-Saclay] Tres variable
                    \\Capacité d'accueil : 378
                    \\Le master mention informatique de l’Université Paris-Saclay forme les chercheurs et les ingénieurs en informatique de demain, capables de faire face aux  grands défis comme les Big Data, l'Intelligence Artificielle, l’Interaction Humain Machine, l'Internet  des Objets, la Cyber-Sécurité ou encore l'Informatique Quantique.
                    \begin{itemize}
                        \item Data Science - Alternance 
                        \item Advanced Networks and Optimization 
                        \item Algorithmique et Modélisation à l'Interface des Sciences 
                        \item Computer \& Network Systems 
                        \item Data Science 
                        \item DataScale : Gestion de données et extraction de connaissances à large échelle 
                        \item Human computer Interaction 
                        \item Ingénierie des réseaux et des systèmes – Voie initiale -continue 
                        \item Ingénierie des réseaux et des systèmes – Voie alternance-apprentissage 
                        \item Master parisien de recherche en Informatique (Mpri) 
                        \item Master parisien de recherche en Informatique (Mpri) - campus orsay 
                        \item Quantum and Distributed Computer Science 
                        \item Sécurité des Contenus, des Réseaux, des Telecommunications et des Systèmes 
                    \end{itemize}
                    $Lien$: \href{https://www.universite-paris-saclay.fr/formation/master/informatique#liste}{Site Université}
                    \\ remarque: Beaucoup de chose, mais paris quoi trop cher, m'enfin a choisir le dernier.
\\
        \item [\color{LightOrangeHaf}Université Paris-VIII] Du 08/02 au 09/07
                    \\Capacité d'accueil : 60
                    \\Le M1 constitue un tronc commun permettant de consolider les bases techniques et méthodologiques et propose des cours plus spécialisés en IA, Big Data et Hypermédia.  Le choix des options est déterminant pour la poursuite des études et doit être fait dès le début du master:
                    \begin{itemize}
                        \item Big Data (BD)
                        \item Conduite de Projets Informatiques
                        \item Ingénierie en Intelligence Artificielle (IIA)
                        \item Technologies de l’Hypermédia
                    \end{itemize}
                    $Lien$: \href{https://www.univ-paris8.fr/-Master-Conduite-de-Projets-Informatiques-}{Site Université}
                    \\ remarque : Paris encores, La conduite de projet a l'aire utile.
\\
        \item [\color{LightOrangeHaf}Université Paris-XIII] Du 15/04 au 30/06
                    \\Capacité d'accueil : 60
                    \\Le parcours EID2 a pour objet l’acquisition de connaissances approfondies dans des domaines particuliers complémentaires : l’informatique décisionnelle et l’extraction de connaissances à partir de données. Cette spécialisation a pour vocation de :
                    \begin{itemize}
                        \item Parcours programmation et logiciels sûrs
                        \item Informatique des données et décisionnel 
                    \end{itemize}
                    $Lien$: \href{https://galilee.univ-paris13.fr/master/master-informatique/}{Site Université}
                    \\ remarque: Le pls me fait grave envie mais apres a paris il y a aussi g\e nial, j'hesite\ldots
\\
        \item [\color{LightOrangeHaf}Université PSL] Du 03/04 au 30/07
                    \\Capacité d'accueil : 115
                    \\Fruit de la collaboration entre Dauphine – PSL, l’École normale supérieure – PSL et MINES Paris – PSL,le master Informatique forme des spécialistes en informatique au sens large, avec une orientation vers les sciences des organisations, les sciences des données et de la décision, l’intelligence artificielle.  Les différents parcours du master préparent aussi bien à la recherche académique qu’aux métiers exercés en entreprise. 
                    \begin{itemize}
                        \item Informatique fondamentale
                        \item Informatique, Décision, Données
                        \item Méthodes Informatiques Appliquées pour la Gestion des Entreprises
                    \end{itemize}
                    $Lien$: \href{https://psl.eu/formation/master-informatique-0}{Site Université}
                    \\ remarque: Paris non non, sinon parours deux.
\\
        \item [\color{LightOrangeHaf}Université Mont Blanc] Du 03/04 au 30/05
                    \\Capacité d'accueil : 32
                    \\Le master informatique permet, en deux ans, à des étudiants titulaires d'une licence d'informatique, d’acquérir les compétences scientifiques et techniques indispensables au développement d’applications numériques.  Cette formation couvre un spectre large avec des enseignements plus appuyés en génie logiciel (conception de systèmes logiciels, architectures logicielles et qualité) et modèles d’interaction (parallélisme, applications et systèmes répartis, architectures orientées services) ainsi qu'en fondement de l’informatique (algorithmique, méthodes formelles, programmation générique).
                    \\$Lien$: \href{https://formations.univ-smb.fr/fr/catalogue/master-XB/sciences-technologies-sante-STS/master-informatique-KHQI8NZ0.html}{Site Université}
        \item [\color{LightOrangeHaf}Université Toulouse] Du 26/04 au 17/05
                    \\Capacité d'accueil : 203
                    \\Le Master Informatique répond aux besoins de l’industrie et de la recherche au niveau du site, mais également aux niveaux national et international. Il aborde, en particulier, les différentes problématiques concernant les pôles de compétitivité de la région (Aerospace Valey tourné vers l’aéronautique, l’espace et les systèmes embarqués, Cancer Bio Santé tourné vers le domaine médical, Agrimip Innovation tourné vers l’agriculture et l’agro-industrie). Par ailleurs, la plupart des parcours du Master Informatique sont en cours de labellisation par l’institut interdisciplinaire d’intelligence artificielle de Toulouse ANITI.
                    \begin{itemize}
                        \item Intelligence Artificielle : Fondements et Applications (IAFA)
                        \item Interaction Homme Machine (IHM)
                        \item Sciences du Logiciel (SDL)
                        \item Systèmes Embarqués et Connectés : Infrastructures et Logiciels (SECIL)
                        \item Computer Science for Aerospace (CSA)
                    \end{itemize}
                    $Lien$: \href{https://departement-informatique.univ-tlse3.fr/master/master-informatique-2021-2026/}{Site Université}
                    \\ remarque: SDL est cool les reste aussi d'ailleur
\\TODO : la sorbone
    \end{enumerate}
\end{document}