\documentclass[addpoints]{exam}
\usepackage{xpatch}
\usepackage{fdsymbol}
\usepackage{xcolor}
\newcommand{\e}{\'{e}}
\printanswers

% Need both to patch both instances (choice, correctchoice) in the oneparchoices environment.
% Unless there is a global replacement that works within an environment. I don't know.
% Replaces \hskip 1em with \hfill
\xpatchcmd{\oneparchoices}{\penalty -50\hskip 1em plus 1em\relax}{\hfill}{}{}
\xpatchcmd{\oneparchoices}{\penalty -50\hskip 1em plus 1em\relax}{\hfill}{}{}
\rhead{I41}
\lhead{CT 2021}
\begin{document}
\begin{center}
\fbox{\fbox{\parbox{5.5in}{\centering
Ce QCM est la reconstituion de la premiere partie du CT de I41, il r\e sulte de la prise de note pendant le CT,ainsi que la consultation des copie. De facto si 95\% des question sont identique a l'original des erreurs sont possible, particulierement pour la correction. Si vous avez un doute n'h\e siter donc pas \`{a} le partager sur $discord$
\\BONNE R\'{E}VISSION !!
\\ pas de correction pour la 18 ni la 4 dsl, mais je serais ravis d'avoir la reponse\ldots}}}
\end{center}
\begin{questions}

\question Qu'elle est la defintion d'une permutation ?
\vspace{3mm}
\\
\fbox{%
  \parbox{\textwidth}{
    {\color{violet}Nous notons$\mathfrak{F} (\sigma )$ l'ensemble des inversion d'une permutation$\sigma$ de $\mathfrak{S}$ }\\
    {\color{violet}$\mathfrak{F} (\sigma ):=\{(i,j)\in[1,n]^2|(i<j)\land (\sigma(i)>\sigma(j))$ }
  }%
}

\question Qu'elle est la definition $formelle$ d'un algorithme ?
\begin{center}
\fbox{%
  \parbox{\textwidth}{
    {\color{violet}Un programme de la machine $RAM$ qui se termine}
  }%
}
\end{center}
\question Rappelez la definition d'un arbre ordonn\'{e} ?
\\
\fbox{%
  \parbox{\textwidth}{
    {\color{violet}Un arbre binaire A=(X,U) de valuation v est appelé arbre partiellement ordonné ssi\\
    $\forall x\in X,\ v(x)\geqslant v(x_g) et v(x)\geqslant v(x_d)$ \\
     ou $x_g$ designe le fils de gauche de x (est resp.)}
  }%
}
\question Soit $A$ et $B$ deux ensemble cod\'{e}s sous forme d'entier de vecteur caract\'{e}ristiques. Donner une expression logique dont la valeur de v\'{e}rit\'{e} est celle de la proposition $A \subset B$ ou $\subset$ designe l'inclusion stricte. On rapelle que $\wedge, \lor, \oplus, \lnot$ d\'{e}signent respectivement le $et, ou, xor, not$ et la $negation$ logique. 
\vspace{3mm}
\\\ldots\ldots\ldots\ldots\ldots\ldots\ldots\ldots\ldots\ldots\ldots\ldots\ldots\ldots\ldots\ldots\ldots\ldots\ldots\ldots\ldots\ldots\ldots\ldots\ldots\ldots\ldots\ldots\ldots\ldots

\question
Calulez $\Theta (n)-\Theta (n)$:

\begin{oneparchoices}
\choice $\Theta(1)$
\choice $1$
\choice $\Theta(0)$
\CorrectChoice $\Theta(n)$
\end{oneparchoices}
\renewcommand\questionlabel{\thequestion.}

\question
$\clubsuit$ On peut trier les num\'{e}ros de t\'{e}l\'{e}phone d'un repertoire de $n$ num\'{e}ros en temps:

\begin{oneparchoices}
\CorrectChoice $O(n)$
\choice $\Theta(log\: n)$
\choice $\Theta(1)$
\choice $0(log\: n)$
\end{oneparchoices}

\question
Qu'elle est la profondeure $p$ d'un arbre binaire equilibre(tout niveau dont la profodeure est <p est plein) qui contient 85 noeud(on rapelle que une feuille est un noeud particulier)

\begin{oneparchoices}
\choice 5
\choice 4
\choice 8
\CorrectChoice 7 
\end{oneparchoices}

\question
$\clubsuit$ Combiens y-a-t'il de $sequence\ binaires$ de longueur n et de poids p ?

\begin{oneparchoices}
\choice $2^n$
\CorrectChoice  $\binom{n}{p}$
\choice $2^p$
\choice $\frac{n!}{(n-p)!}$
\\
\choice $2^{n-p}$
\CorrectChoice  $\frac{n!}{p!(n-p)!}$
\choice $\frac{p!}{n!}$
\choice $p \times n$
\end{oneparchoices}

\question
Soit $k\in \mathbb{N}\backslash\{0\}$L'\e criture en base 4 de $16^k-1$ contient combiens de chiffre ?

\begin{oneparchoices}
\choice k chiffre
\choice k-4 chiffre
\choice k+1 chiffre
\CorrectChoice k+2 chiffre 
\end{oneparchoices}

\question
Peut-on r\e organiser le contenu des tiroirs d'une commode (une pile de tiroirs) de la manni\`{e}re de son choix en ne faisant que des \e changes entre deux tiroirs adjacents ?

\begin{oneparchoices}
\choice Non
\CorrectChoice Oui 
\choice Cela d\e pend du nombre de tiroirs
\end{oneparchoices}

\question
$\clubsuit$ Soit $n\in \mathbb{N} $. Combiens existe t'il de $transposition$(permutation qui \e change deux element et laissent les $n$-2 autres fixe) dans le groupe $\mathfrak{S}_n$ ? 

\begin{oneparchoices}
\CorrectChoice $\frac{n(n-1)}{2}$
\choice $\frac{1}{2}$
\choice $\sum_{i=1}^{n}i$
\choice $\frac{n(n+1)}{2}$
\\
\choice  $n$
\CorrectChoice $\binom{n}{2}$
\choice $2n$
\choice $n^2$
\end{oneparchoices}

\question
$\clubsuit$ Soit E=$\{x_0,x_1\ldots x_7\}$ un ensemble et $A$ et $B$  deux partie de $E$ cod\e\  par le vecteur carateristiques entier. Quel(s) entier(s) code(nt) la partie($A\cap B$)\textbackslash $\{x_1,x_4\}$ ? On rapelle que $\wedge, \lor, \oplus, \lnot$ d\'{e}signent respectivement le $et, ou, xor, not$ et la $negation$ logique.

\begin{oneparchoices}
\choice $A\wedge B\wedge 18$
\choice $\lnot (\lnot A\lor 18\lor \lnot B)$
\choice $A\lor B\lor 16$
\\
\choice $(A\lor B)\land 18$
\choice $(A\land B)\land \lnot 18$
\CorrectChoice $(A\land B)\lor \lnot 16$
\end{oneparchoices}

\question
Soit $n$ un entier naturel. Combien de chiffre l'entier $n!$ contient t'il  dans sont ecriture dans une base donn\e e ?

\begin{oneparchoices}
\choice $\log n$
\CorrectChoice $\Theta (n\ \log n)$
\choice $\Omega (n^2)$
\choice $\frac{n(n+1)}{2}$
\choice $\Theta (\log n(\log n+1))$
\end{oneparchoices}

\question
La recherche dichotomique dns une liste d'element $trie$ s'effectue en temps:

\begin{oneparchoices}
\choice Lin\e aire
\choice Quasi-ineaire
\CorrectChoice Logarithmique
\choice Quadratique
\end{oneparchoices}

\question
Le nombre de produit de l'algorithme d'H\"orner pour \e valuer une fontion de degr\e \ $n$ est:

\begin{oneparchoices}
\CorrectChoice Lin\e aire
\choice Quasi-lin\e aire
\choice Logarithmique
\choice Quadratique
\end{oneparchoices}

\question
$\clubsuit$ La complexit\e de l'algo $Entasser$ qui transforme une liste en tas est:

\begin{oneparchoices}
\CorrectChoice Lin\e aire
\choice Quasi-ineaire
\choice Logarithmique
\\
\CorrectChoice $O(n)$
\choice $0(log\: n)$
\choice Quadratique
\end{oneparchoices}

\question
$\clubsuit$ La complexit\e \ du trie $Lexicographique$ est :

\begin{oneparchoices}
\CorrectChoice Lin\e aire
\choice Quasi-lin\e aire
\choice Logarithmique
\\
\CorrectChoice $O(n)$
\choice $0(log\: n)$
\choice Quadratique
\end{oneparchoices}


\question
Calculer la  somme $\sum_{p=2}^{7} \binom{7}{p}$

\begin{oneparchoices}
\choice 123
\choice 115
\choice 60
\choice 120
\choice 67

\end{oneparchoices}
\question
Qu'elle est l'expression postfixe de l'expression suivante:
\begin{center}
    $[(5-2)\times(x+3)]+(5-2)\times2 $ ?
\end{center}


\begin{oneparchoices}
\CorrectChoice 5,2,-,x,3,+,*,5,2,-,2,*,+
\choice 5,2,-,x,3,+,*,+,5,2,-,2,*
\\
\choice 5,-,2,*,x,3,+,5,-,2,*,2
\choice 5,2,*,x,3,2,2,*,+,-,+
\end{oneparchoices}



\question
Quel est l'indice $q$ renvoy\e  par l'algorithme  de partionement du trie rapide pour la liste $L$=[3,5,5,1,3,2,3,7,1,4] (l'indexation commence a 1)?

\begin{oneparchoices}
\choice 1
\choice 2
\choice 4
\CorrectChoice 5
\choice 6
\end{oneparchoices}




\end{questions}
\end{document}
