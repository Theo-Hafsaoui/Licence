\documentclass[French,Hoar.tex]{subfiles}

\begin{document}

  \section{La logique de Hoare}
  %src https://www.cs.cmu.edu/~aldrich/courses/15-819O-13sp/resources/hoare-logic.pdf
  \subsection{Presentation}
    Le but de la \emph{Logique de Hoare} est de permettre un systeme formelle de raisonement 
    sur la justesse d'un programme
    Plus concrettement comme chacun le sait il existe des systeme critique pour lequel l'incertitude
    n'est pas acceptable.
    Cette logique est baser sur les \textbf{\color{nred}{ triplet de Hoare}},
    ce triplet de la forme $\{P\} S \{Q\}$ est respectivment composer de :
    %TODO ajouter plus d'explication
    \begin{itemize}
      \item[\ding{227}] $P$ Une precondition
      \item[\ding{227}] $S$ Le programme
      \item[\ding{227}] $Q$ Une postcondition
    \end{itemize}
    Les precondition est postcondition sont deux assertion appartenant a la \href{http://zanotti.univ-tln.fr/MD/MD-Ensembles.html#pr%C3%A9dicats}{logique des predicats}.
  \subsection{M\e thode}

\end{document}
