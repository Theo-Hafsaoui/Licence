\documentclass[French,Hoar.tex]{subfiles}

\begin{document}

  \section{Le Quicksort}
  \subsection{Presentation}
  Le \emph{Quicksort} ou trie rapide dans la langue de Molière, est un algorithme de trie qui est \ldots \ rapide.
  De manière Plus concrète il s'agit d'un trie qui se rapproche des limites possible d'un trie comparatifs sans
  propriete connue.
 %TODO Rajouter source zannoti http://zanotti.univ-tln.fr/ALGO/II/BorneTrisComparatifs.html
  %add figure
  \begin{center}
  \large{\color{Nred}      }
% src:https://tex.stackexchange.com/questions/573966/copy-the-merge-sort-recursion-tree-from-tikz-example-with-forest-package
  \begin{tikzpicture}[level/.style={sibling distance=30mm/#1}]
  \node [color=Nblack] (z){$[38,27,43,3,9,82,10]$}
    child {node [color=nred] (a) {$[39,27,43,3]$}
      child {node [color=nred] (b) {$[39,27]$}
        child {node {$\vdots$}
          child {node [color=nred] (d) {$39$}}
          child {node [color=nred] (e) {$27$}}
        } 
        child {node {$\vdots$}}
      }
      child {node [color=Nblack] (g) {\ldots}
        child {node {$\vdots$}}
        child {node {$\vdots$}}
      }
    }
    child {node [color=nblue] (j) {$[9,82,10]$}
      child {node [color=Nblack] (k) {$\ldots$}
        child {node {$\vdots$}}
        child {node {$\vdots$}}
      }
    child {node [color=nblue] (l) {$[82,10]$}
      child {node {$\vdots$}}
      child {node (c){$\vdots$}
        child {node [color=nblue] (o) {$82$}}
        child {node [color=nblue] (p) {$10$}
          child [grow=right] {node (q) {  } edge from parent[draw=none]
            child [grow=right] {node (q) {  } edge from parent[draw=none]
              child [grow=up] {node (r) {  } edge from parent[draw=none]
                child [grow=up] {node (s) {  } edge from parent[draw=none]
                  child [grow=up] {node (t) {  } edge from parent[draw=none]
                    child [grow=up] {node (u) {  } edge from parent[draw=none]}
                  }
                }
              }
              child [grow=down] {node (v) {  }edge from parent[draw=none]}
            }
          }
        }
      }
    }
  };
  \end{tikzpicture}

  \end{center}

  Ce trie repose sur le principe du diviser pour mieux regner. Si en politique cela consiste a augmenter le nombre
  de candidats d'oposition, en informatique cela consiste en trois etapes:
  \begin{enumerate}
    \item \textbf{\color{Nred}Diviser}, on divise l'instance en plus petite instance, en generale en separant en deux ce qui
      explique le retour recurent du $\log_2$ dans la complexiter de cette famille d'algorithme.
 %TODO Lien vers la dite famille
    \item \textbf{\color{Nred}Regner},cette partie est surement la plus evidente car c'est la que l'ont le trie.
    \item \textbf{\color{Nred}Reunir}, Enfin reunie afin d'obtenir notre resultat final, c'est d'ailleur cette
    partie qui donne le charactere linaire des trie \emph{"divide and conquere"}
  \end{enumerate}
    Le trie dans l'exemple n'est pas le trie rapide, mais le trie fusion un trie baser lui aussi sur le \emph{Divide and 
    conquere}, il est a noter que la figure ne reprensente que la phase de division et pas la fusion elle meme, en effet celle 
    ci est specifique au merge sort.

%rajouter source dalinar
  \subsection{Principe et implementation}
  Le trie rapide,comme dit precedament, repose sur le \emph{divide and conquere} plus particulierment autour d'un \textbf{\color{nred}pivot}.
  Ce pivot possede neanmoins une propriete interesante, a la suite d'un \emph{partitionment} tout les element a gauche lui sont inferieur et respectivment, ce que l'ont peut resumer par :
  $$
    \forall (i,j)\in[p,q]\times[q+1,r],\quad L[i] \leq L[j].
  $$
  Cette propriete implique une infomation inportante, \textbf{tout les partition sont trier}, ainsi et de manière naturelle.
  on peut juste ce contenter de reproduire cette procedure de maniere recursive.
  %figure de la video
  %explication
  \lstinputlisting[style=C]{./src/Quicksort.c}
  \subsection{Complexite}
  De manière étonnante pour un trie utiliser aussi fréquemment sont pire cas est en $\Theta(n^2)$, neanmoins il existe
  de nombreuse variante de ce trie qui résolve ce problème. Un fix rapide est par exemple le choix du pivot de maniere al\e atoire


\end{document}
