\documentclass[French,Hoar.tex]{subfiles}

\begin{document}

  \section{Le Quicksort}
  \subsection{Presentation}
  Le \emph{Quicksort} ou trie rapide dans la langue de moliere, est un algorithme de trie qui est \ldots \ rapide.
  De maniere plus concrete il s'agit d'un trie qui se raproche des limites possible d'un trie comparatife sans
  propriete connue.
 %TODO Rajouter source zannoti http://zanotti.univ-tln.fr/ALGO/II/BorneTrisComparatifs.html
  %add figure

  Ce trie repose sur le principe du diviser pour mieux regner. Si en politique cela consiste a augmenter le nombre
  de candidats d'oposition, en informatique cela consiste en trois etapes:
  \begin{enumerate}
    \item \textbf{\color{Nred}Diviser}, on divise l'instance en plus petite instance, en generale en separant en deux ce qui
      explique le retour recurent du $\log_2$ dans la complexiter de cette famille d'algorithme.
 %TODO Lien vers la dite famille
    \item \textbf{\color{Nred}Regner},cette partie est surement la plus evidente car c'est la que l'ont le trie.
    \item \textbf{\color{Nred}Reunir}, Enfin reunie afin d'obtenir notre resultat final, c'est d'ailleur cette partie qui donne
    le charactere linaire des trie \emph{"divide and conquere"}

    Le trie dans l'exemple n'est pas le trie rapide, mais le trie fusion. 
%rajouter source dalinar
  \end{enumerate}

\end{document}
